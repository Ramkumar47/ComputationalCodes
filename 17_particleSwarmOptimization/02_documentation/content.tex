
\begin{frame}
    \frametitle{PSO Algorithm introduction}
    \begin{itemize}
        \item A non-gradient, population-based algorithm for optimization
        \item Inspired from the nature: flock of birds in search of food etc...
        \item It uses a number of particles whose position and velocity will
            depend on the personal and global optimum locations
        \item Velocity of each particle is calculated/updated on each iteration
            with below equation
            \begin{align*}
                \bar{v}^{(k+1)} = w \bar{v}^{(k)} + C_1 r_1^{(k)} \left(\bar{p}_{best} - \bar{x}^{(k)}\right) + C_2 r_2^{(k)} \left(\bar{g}_{best} - \bar{x}^{(k)}\right)
            \end{align*}
            \(w,C_1,C_2\) are constants, and \(r_1,r_2\) are random values between 0 to 1 for each iteration
        \item The position of each particle is updated using below equation
            \begin{align*}
             \bar{x}^{(k+1)} = \bar{x}^{(k)} + \bar{v}^{(k+1)}
            \end{align*}
    \end{itemize}
\end{frame}

% -----------------------------------------------------------------------------

\begin{frame}
    \frametitle{Program key points}
    \begin{itemize}
        \item Code was developed using Python programming language
        \item Each particle was treated as an object instance from the main
            Particle class definition
        \item Main script file and a separate inputFile.py were developed to
            induce undisturbance to the main file
        \item Objective function and its sampling ranges, along with other inputs
            like number of particles, can be defined in the inputFile.py file
        \item Program will output a final\_output.csv file that will contain
            the final position and velocity information of all particles at
            the end of iteration
        \item As optional output features, the history of each particle and
            the contour outputs can also be obtained from the program
    \end{itemize}
\end{frame}

%------------------------------------------------------------------------------

\begin{frame}
    \frametitle{Test Optimization functions}
    \tiny
    The following test functions were used for demonstration
    \begin{itemize}
        \item Deformed egg carton function
            \begin{align*}
                f(x,y) = (x-3.14)^2 + (y-2.2)^2 +\sin(3 x+1.41) + \sin(4 y - 1.73) \\
                0 \le x,y \le 5
            \end{align*}
        \item Beale function
            \begin{align*}
                f(x,y) = (1.5-x+xy)^2+(2.25-x+x y^2)^2 + (2.625 -x+xy^3)^2 \\
                -4.5 \le x,y \le 4.5
            \end{align*}
        \item Himmelblau function
            \begin{align*}
                f(x,y) = (x^2+y-11)^2+(x+y^2-7)^2 \\
                -5 \le x,y \le 5
            \end{align*}
        \item Three hump camel function
            \begin{align*}
                f(x,y) = 2 x^2 -1.05x^4+\frac{x^6}{6}+x y +y^2 \\
                -5 \le x,y \le 5
            \end{align*}
    \end{itemize}
\end{frame}

\begin{frame}
    \frametitle{Test Optimization functions Contd. }
    \tiny
    \begin{itemize}
        \item Egg holder function
            \begin{align*}
                f(x,y) = -(y+47)\sin\sqrt{|\frac{x}{2} + (y+47)|} - x\sin\sqrt{|x-(y+47)|} \\
                -512 \le x,y \le 512
            \end{align*}
        \item McCormick function
            \begin{align*}
                f(x,y) = \sin(x+y) + (x-y)^2 -1.5 x + 2.5 y + 1 \\
                -1.5 \le x \le 4, -3 \le y \le 4
            \end{align*}
    \end{itemize}
\end{frame}
%------------------------------------------------------------------------------
\begin{frame}
    \frametitle{Output from program}
    The following items were output from program for each test function and
    are present in its own directories
            \vspace{1cm}
    \begin{itemize}
        \item final particle data such as velocity and position \textit{final\_data.csv}
            \vspace{1cm}
        \item Animated video output \textit{output.mp4}
    \end{itemize}
\end{frame}
