\section{Python code structure}
\par{}
In each of the source code that was provided with this documentation,
there will be 3 python scripts, each handling specific part of the program.
They are as listed below.
\begin{itemize}
    \item script\_main.py
    \item inputData.py
    \item customLayers.py
\end{itemize}

\par{}
The \textit{script\_main.py} is the main script that constructs the network and
has snippets related to physics information. \textit{inputData.py} script
contains the snippets that determine the input parameters, such as number of
layers and neurons on them, whether to save/load weights and learning rate
adaptation code. Finally, \textit{customLayers.py} contains the class
definition of a neural network layer that has everything related to layers such
as activation function, its derivative and forward evaluation. Activation function
is hard coded for now, and later may be made generalizable. \\

\par{}
The code was made as generalizable in terms of having number of neurons/layers
for now, and the code is for fully connected dense network. Later, the future
work may include custom codes for other network types when the need arises. \\


\section{Conclusion and future works}
\par{}
I conclude the present work by stating that a high level of understanding and
knowledge gained in the perspective of mathematics and programming of neural
networks, that opened gateways for other methods of using neural networks,
such as optimizing input values itself based on loss to find optimal
values etc... The present work is limited to the generalization code of
fully connected dense network. And, this code may be extended for other
scientific computation uses such as Neural ODEs.
